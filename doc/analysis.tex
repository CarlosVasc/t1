\section{Theoretical Analysis}
\label{sec:analysis}

Por obséquio, tenha a gentileza de funcionar, senhor

In this section, the circuit shown in Figure~\ref{fig:rc} is analysed
theoretically, in terms of its time and frequency responses.

\section{Valores octave}



\begin{table}[h]
  \centering
  \begin{tabular}{|l|r|}
    \hline    
    {\bf Name} & {\bf Value [A or V]} \\ \hline
    V1 & -0.249911\\ \hline 
V2 & -0.778383\\ \hline 
V3 & 3.700284\\ \hline 
V4 & -8.073949\\ \hline 
V5 & -7.067894\\ \hline 
V6 & -5.113245\\ \hline 
V7 & -0.214558\\ \hline 
Ivc & 0.000036\\ \hline
  \end{tabular}
  \caption{Operating point. A variable preceded by @ is of type {\em current}
    and expressed in Ampere; other variables are of type {\it voltage} and expressed in
    Volt.}
  \label{tab:op}
\end{table}


\begin{table}[h]
  \centering
  \begin{tabular}{|l|r|}
    \hline    
    {\bf Name} & {\bf Value [A or V]} \\ \hline
    Ia & -0.000246\\ \hline 
Ib & -0.000257\\ \hline 
Ic & 0.000968\\ \hline 
Id & 0.001004\\ \hline
  \end{tabular}
  \caption{Operating point. A variable preceded by @ is of type {\em current}
    and expressed in Ampere; other variables are of type {\it voltage} and expressed in
    Volt.}
  \label{tab:op}
\end{table}



\begin{table}[h]
  \centering
  \begin{tabular}{|l|r|}
    \hline    
    {\bf Name} & {\bf Value [A or V]} \\ \hline
    Ir3 & 0.000011\\ \hline 
Ir4 & -0.001214\\ \hline 
Ir5 & 0.001261\\ \hline 
Ivc & 0.000036\\ \hline
  \end{tabular}
  \caption{Operating point. A variable preceded by @ is of type {\em current}
    and expressed in Ampere; other variables are of type {\it voltage} and expressed in
    Volt.}
  \label{tab:op}
\end{table}

